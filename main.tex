%%
%
% ARQUIVO: main.tex
%
% VERSÃO: 1.1
% DATA: Junho de 2016
% AUTOR: Coordenação PPgSC
% 
%  Arquivo tex principal do documento de Dissertação.
%  Este arquivo SÓ PRECISA SER MODIFICADO NA PARTE DE CONTEÚDO:
%
%    a. colocar um \include{•} para cada capítulo da Dissertação.
%
%%

% -----
% CLASSE DO DOCUMENTO DA DISSERTAÇÃO
% -----
\documentclass{dissertacao}

% -----
% PACOTES LATEX USADOS NO DOCUMENTO DA DISSERTAÇÃO
% -----
\usepackage[brazilian]{babel}
\usepackage[utf8]{inputenc}
\usepackage[T1]{fontenc}

\usepackage{amsmath}
\usepackage{graphicx}
\usepackage{tabularx}
\usepackage{float}
\usepackage{color}
\usepackage{amsfonts,amssymb}
\usepackage[authoryear]{natbib}

\usepackage{enumitem}
\usepackage{rotating}
\usepackage{lipsum}
\usepackage{lastpage}
\usepackage{stringstrings}
\usepackage{pgffor}
\usepackage{setspace}

% -----
% MARGENS DO DOCUMENTO DA DISSERTAÇÃO
% -----
\usepackage{geometry}
\geometry{
	a4paper,
	total={210mm,297mm},
	left=25mm,
	right=25mm,
	top=25mm,
	bottom=30mm,
	textwidth=160mm,
	textheight=242mm,
	headheight=0mm,
	headsep=0mm,
}

% -----
% DECLARAÇÕES AUXILIARES PARA REFERÊNCIAS
%
%  Diferencia \citet e \citep de acordo com a NBR 10520:2002
% -----
\DeclareRobustCommand{\NATand}{;}
\DeclareRobustCommand{\NATetal}{et~al.}
\makeatletter
\renewcommand{\NAT@nmfmt}[1]{%
  \ifNAT@swa\expandafter\MakeUppercase
  \else\DeclareRobustCommand{\NATand}{ e}\expandafter\@firstofone\fi{{\NAT@up #1}}%
}
\makeatother

% -----
% AMBIENTE DE FIGURAS DA DISSERTAÇÃO
%
%  A classe do documento está configurada SOMENTE para figuras no formato EPS.
%  Logo, use PREFERENCIALMENTE este tipo de arquivo.
%
%    a. os arquivos das figuras devem estar no diretório 'img'
% -----
\graphicspath{{./img/}}

% -----
% INÍCIO DO DOCUMENTO DA DISSERTAÇÃO
% -----
\begin{document}

% -----
% PARTE PRÉ-TEXTUAL DA DISSERTAÇÃO
%
% Alterar o CONTEÚDO dos arquivos siglas.tex E pre-texto.tex
% -----
%%
%
% ARQUIVO: dados-dissertacao.tex
%
% VERSÃO: 1.0
% DATA: Novembro de 2015
% AUTOR: Coordenação PPgSC
% 
%  Arquivo tex com os dados acerca da Dissertação e da defesa.
%
%   nos campos que definem nomes (autor; orientador; co-orientador; membros da banca)
%   É PRECISO usar os COMANDOS LaTeX para acentuação, conforme abaixo:
%
%         \'a - á || \`a - à || \~a - ã || \^a - â 
%         \'e - é || \^e - ê || \'i - í 
%         \'o - ó || \~o - õ || \^o - ô 
%         \'u - ú || \"u - ü
%
%%

%%% AUTOR DA DISSERTAÇÃO (Nome completo)
\autor{Seu Nome Completo}

%%% POSTO DO AUTOR DA DISSERTAÇÃO
% ---
%  se o autor é CIVIL, REMOVA ESTA LINHA
%  se o autor é MILITAR, coloque CORRETAMENTE o POSTO aqui
% ---
%\postoautor{1 Ten}

%%% TITULO DA DISSERTAÇÃO
\titulo{Título Completo da Dissertação}

%%% DATA DA DEFESA (formato {dd}{Mmmmm}{aaaa})
\datadefesa{31}{Fevereiro}{2015}

%%% ORIENTADOR DA DISSERTAÇÃO
% ---
%  CAMPO 1: P (para Prof.); PA (para Profa.); ou qualquer coisa (inclusive VAZIO) - o que for escrito aparecerá no documento
%  CAMPO 2: Nome completo
%  CAMPO 3: D (para D.Sc.); P (para Ph.D.); ou qualquer coisa (inclusive VAZIO) - o que for escrito aparecerá no documento
%  CAMPO 4: Instituição (com "do / da")
% ---
\orientador{PA}{Nome Completo do Orientador}{D}{do IME}

%%% CO-ORIENTADOR DA DISSERTAÇÃO
% ---
%  se não houver co-orientador, REMOVA ESTA LINHA
%  preenchimento idêntico a \orientador{}{}{}{}
% ---
\coorientador{P}{Nome Completo do Co-orientador}{P}{do IME}

%%% NÚMERO DA ENTRADA DA BIBLIOTECA (pegar na Biblioteca do IME)
\biblioref{004.69}{S586e}

%%% PALAVRAS-CHAVES DA DISSERTAÇÃO
% ---
%  devem ser separadas por vírgula e deve ter pelo menos uma
% ---
\palavraschaves{Palavra 01, Palavra 02, Palavra 03}

%%% OUTROS MEMBROS DA BANCA DA DISSERTAÇÃO
% ---
%  aceita até mais 05 membros (de membrobancaI até membrobancaV)
%    a. preemcher sucessivamente a partir de membrobancaI
%    b. REMOVER as definições não necessárias
%
%  cada membro tem preenchimento idêntico a \orientador{}{}{}{}
% ---
\membrobancaI{}{Nome do Membro da Banca 1}{}{da PUC-Rio}
\membrobancaII{P}{Nome do Membro da Banca 2}{P}{do Massachussets Institute of Technology}
%\membrobancaIII{}{Nome do Membro da Banca 3}{}{da COPPE/UFRJ}
%\membrobancaIV{}{Nome do Membro da Banca 4}{}{da UNIRIO}
%\membrobancaV{}{Nome do Membro da Banca 5}{}{da UERJ}

%%
%
% ARQUIVO: pre-texto.tex
%
% VERSÃO: 1.0
% DATA: Novembro de 2015
% AUTOR: Coordenação PPgSC
% 
%  Arquivo tex para a criação da parte pré-textual da Dissertação.
%
%%


% -----
% PÁGINA DE CAPA DA DISSERTAÇÃO
% -----
\makecapa

% -----
% PÁGINA DE TÍTULO DA DISSERTAÇÃO
% -----
\prepareadvisors
\maketitle

% -----
% PÁGINA DE CRÉDITOS DA DISSERTAÇÃO
% -----
\makecredits

% -----
% PÁGINA DE FOLHA DE ASSINATURAS
% -----
\preparemembers
\approvalpage

% -----
% PÁGINA DE DEDICATÓRIA (OPCIONAL, ie. pode remover toda a página)
% -----
%%% DEDICATÓRIA - PREENCHER...
\dedicatoria{%
Ao Instituto Militar de Engenharia, alicerce da minha formação e aperfeiçoamento.
}%
\makededication

% -----
% PÁGINA DE AGRADECIMENTOS (OPCIONAL, ie. pode remover toda a página)
% -----
%%% AGRADECIMENTOS - PREENCHER...
\agradecimentos{%
Agradeço a todas as pessoas que me incentivaram, apoiaram e possibilitaram esta oportunidade de ampliar meus horizontes. \\
\indent
Meus familiares, cônjuge e mestres.\\
\indent
Em especial ao meu Professor Orientador Dr. Antonio José Reis e ao Professor Co-orientador Dr. Joel Duarte Silva, por suas disponibilidades e atenções.
}%
\makethanks

% -----
% PÁGINA DE EPÍGRAFE (OPCIONAL, ie. pode remover toda a página)
% -----
%%% EPÍGRAFE - PREENCHER...
\epigrafe{%
Sem publicação, a ciência é morta.
}%
\autorepigrafe{%    %% Se não tem autor, coloque "Anônimo"
Gerard Piel
}%
\makeepigraph

% -----
% PÁGINA DE SUMÁRIO
% -----
\tableofcontents

% -----
% PÁGINAS DE LISTAS DE FIGURAS E DE TABELAS
% se a Dissertação não possui figuras e/ou tabelas, REMOVA O COMANDO CORRESPONDENTE
% -----
\listoffigures
\listoftables

% -----
% PÁGINA DE LISTA DE SIGLAS
% se a Dissertação não possui siglas, REMOVA TODA A PÁGINA
% -----
%%% SIGLAS - PREENCHER...
\acronimo{LA}{Los Angeles}
\acronimo{NY}{New York}
\acronimo{PRODASEN}{Centro de Informática e Processamento de Dados do Senado Federal}
\acronimo{UN}{United Nations}

\listofnicks

% -----
% PÁGINA DE LISTA DE ABREVIATURAS
% se a Dissertação não possui abreviaturas ou símbolos, REMOVA TODA A PÁGINA
% -----
%%% ABREVIATURAS - PREENCHER...
\abreviatura{Ja}{jacobiano}
\abreviatura{JS}{fluxo secundário (difusivo)}
\abreviatura{M}{número de Mach}

%%% SÍMBOLOS - PREENCHER...
\simbolo{$\Phi$}{termo de dissipação viscosa}
\simbolo{$\Gamma$}{coeficiente de difusão efetivo}
\simbolo{$\alpha$}{fator de sub-relaxação}
\simbolo{$\phi$}{variável dependente da equação diferencial geral}

\listofsymbols

% -----
% PÁGINA DE RESUMO
% -----
%%% RESUMO - PREENCHER...
\resumo{%
At vero eos et accusamus et iusto odio dignissimos ducimus qui blanditiis praesentium voluptatum deleniti atque corrupti quos dolores et quas molestias excepturi sint occaecati cupiditate non provident, similique sunt in culpa qui officia deserunt mollitia animi, id est laborum et dolorum fuga. Et harum quidem rerum facilis est et expedita distinctio. Nam libero tempore, cum soluta nobis est eligendi optio cumque nihil impedit quo minus id quod maxime placeat facere possimus, omnis voluptas assumenda est, omnis dolor repellendus.\\
\indent
Temporibus autem quibusdam et aut officiis debitis aut rerum necessitatibus saepe eveniet ut et voluptates repudiandae sint et molestiae non recusandae. Itaque earum rerum hic tenetur a sapiente delectus, ut aut reiciendis voluptatibus maiores alias consequatur aut perferendis doloribus asperiores repellat.
}%
\makeresumo

% -----
% PÁGINA DE ABSTRACT
% -----
%%% ABSTRACT - PREENCHER...
\abstract{%
\lipsum[1]
}%
\makeabstract


\parindent 0.75cm

% -----
% PARTE DE CONTEÚDO DA DISSERTAÇÃO
%
%  Escrever cada capitulo da Dissertação em um arquivo .tex separado.
%  Adicionar os arquivos .tex à Dissertação com comando \include{•}
% -----
\include{cap-01}

% -----
% PARTE DE REFERÊCIAS BIBLIOGRÁFICAS DA DISSERTAÇÃO
%
%  As referências da Dissertação devem estar no arquivo refs.bib
%  Devem seguir o formato bibtex - ver Manual-Referencias.pdf para mais detalhes.
% -----
\bibliographystyle{dissertacao}
\bibliography{refs}

% -----
% PARTE DE APÊNDICE DA DISSERTAÇÃO
%
%  Se a Dissertação não tiver apêndices REMOVER AS LINHAS ABAIXO
%  Adicionar os arquivos .tex de apêndice à Dissertação com comando \include{•}
% -----
\inappendix
\include{apendice}
\outappendix

% -----
% PARTE DE ANEXO DA DISSERTAÇÃO
%
%  Se a Dissertação não tiver anexos REMOVER AS LINHAS ABAIXO
%  Adicionar os arquivos .tex de anexo à Dissertação com comando \include{•}
% -----
\inannex
\include{anexo}
\outannex

% -----
% FIM DO DOCUMENTO DA DISSERTAÇÃO
% -----
\label{theend}
\end{document}
