%% abtex2-modelo-slides.tex, v-1.0 gfabinhomat
%% Copyright 2012-<COPYRIGHT_YEAR> by abnTeX2 group at http://www.abntex.net.br/ 
%%
%% This work may be distributed and/or modified under the
%% conditions of the LaTeX Project Public License, either version 1.3
%% of this license or (at your option) any later version.
%% The latest version of this license is in
%%   http://www.latex-project.org/lppl.txt
%% and version 1.3 or later is part of all distributions of LaTeX
%% version 2005/12/01 or later.
%%
%% This work has the LPPL maintenance status `maintained'.
%% 
%% The Current Maintainer of this work is Fábio Rodrigues Silva, 
%% member of abnTeX2 team, led by Lauro César Araujo. 
%% Further information are available on 
%% http://www.abntex.net.br/
%%
%% This work consists of the files abntex2-modelo-slides.tex, 
%% abntex2-modelo-references.bib and abntex2-modelo-marca.pdf
%%
%% Modelo desenvolvido por Fábio Rodrigues Silva (gfabinhomat@gmail.com)
%% Mais informações podem ser obtidas no guia do usuário Beamer 
%% (http://linorg.usp.br/CTAN/macros/latex/contrib/beamer/doc/beameruserguide.pdf)
%% Informações rápidas podem ser acessadas em http://en.wikibooks.org/wiki/LaTeX/Presentations


% Apresentações em widescreen. Outros valores possíveis: 1610, 149, 54, 43 e 32.
% Por padrão, as apresentações são no formato 4:3 (sem o aspectratio).
\documentclass[aspectratio=169]{beamer}	 	

\usetheme{Pittsburgh}
\usecolortheme{default}
\usefonttheme[onlymath]{serif}			% para fontes matemáticas
% Enconte mais temas e cores em http://www.hartwork.org/beamer-theme-matrix/ 
% Veja também http://deic.uab.es/~iblanes/beamer_gallery/index.html

% Customizações de Cores: fg significa cor do texto e bg é cor do fundo
\setbeamercolor{normal text}{fg=black}
\setbeamercolor{alerted text}{fg=red}
\setbeamercolor{author}{fg=blue}
\setbeamercolor{institute}{fg=blue}
\setbeamercolor{date}{fg=green}
\setbeamercolor{frametitle}{fg=red}
\setbeamercolor{framesubtitle}{fg=brown}
\setbeamercolor{block title}{bg=blue, fg=white}		%Cor do título
\setbeamercolor{block body}{bg=gray, fg=darkgray}	%Cor do texto (bg= fundo; fg=texto)

% ---
% PACOTES
% ---
\usepackage[alf]{abntex2cite}		% Citações padrão ABNT
\usepackage[brazil]{babel}		% Idioma do documento
\usepackage{color}			% Controle das cores
\usepackage[T1]{fontenc}		% Selecao de codigos de fonte.
\usepackage{graphicx}			% Inclusão de gráficos
\usepackage[utf8]{inputenc}		% Codificacao do documento (conversão automática dos acentos)
\usepackage{txfonts}			% Fontes virtuais
% ---

% --- Informações do documento ---
\title{Modelo de apresentação de slides com Beamer e abnTeX2}
\author{Fábio Rodrigues Silva}
\institute{Universidade do Brasil
	    \par
	    Faculdade de Arquitetura da Informação}
\date{\today, v<VERSION>}
% ---

% ----------------- INÍCIO DO DOCUMENTO --------------------------------------
\begin{document}

% ----------------- NOVO SLIDE --------------------------------
\begin{frame}

\begin{minipage}{1\linewidth}
  \centering
  \begin{tabular}{cc}
    \begin{tabular}{c}
      \includegraphics[width=3.0cm]{abntex2-modelo-img-marca.pdf}
    \end{tabular}
    &
    \begin{tabular}{c}
      \textbf{Universidade do Brasil} \\ \textbf{Faculdade de Arquitetura da Informação}
    \end{tabular}
  \end{tabular}
\end{minipage}

\titlepage

\end{frame}

% ----------------- NOVO SLIDE --------------------------------
\begin{frame}{Sumário}
\tableofcontents
\end{frame}

% ----------------- NOVO SLIDE --------------------------------
\section{Introdução}

\begin{frame}{Introdução}

Sinta-se convidado a participar do projeto abnTeX! Acesse o site do projeto em
\url{http://www.abntex.net.br/}. Também fique livre para conhecer,
estudar, alterar e redistribuir o trabalho do ABNTEX, desde que os arquivos
modificados tenham seus nomes alterados e que os créditos sejam dados aos
autores originais, nos termos da ``The LATEX Project Public
License''\ \url{http://www.latex-project.org/lppl.txt} \cite[p. 31]{abntex2modelo}.

\end{frame}

% ----------------- NOVO SLIDE --------------------------------
\section{Fontes a serem consultadas}
\begin{frame}
\frametitle{Público-Alvo}
\framesubtitle{Usuários já iniciados ao Beamer}

\begin{block}{Título}
 Este modelo foi preparado como uma aplicação do uso do pacote abnTeX2 com o Beamer.
\end{block}

\begin{itemize}
 \item Alguns comandos são explicados no modelo TEX. \pause
 
 \item Para maiores informações, consulte o guia do usuário Beamer 
 (\url{https://www.ctan.org/pkg/beamer})\pause
 
 \item Para alterar o tema e as cores, consulte 
 \url{http://deic.uab.es/~iblanes/beamer_gallery/index.html}
 
 \item Consulte também \url{http://www.hartwork.org/beamer-theme-matrix/}
\end{itemize}

\end{frame}

% ----------------- NOVO SLIDE --------------------------------
\begin{frame}{CTAN}

Visite com frequência a página \url{http://www.ctan.org/}. 
Use-a como um guia de orientações gerais.
\vspace{0.7cm}

Outras fontes a serem consideradas:
\begin{enumerate}
 \item \url{http://www.latex-project.org/}
 \item \url{http://www.tex-br.org/}
 \item \url{http://latexbr.blogspot.com.br/}
 \item \url{http://tex.stackexchange.com/}
 \item \url{http://www.tug.org/}
\end{enumerate}

\end{frame}

% ----------------- NOVO SLIDE --------------------------------
\begin{frame}

\begin{figure}
  \centering
  \includegraphics[scale=1.0]{abntex2-modelo-img-marca.pdf}
  \caption{Marca abnTeX2. Fonte: \url{http://www.abntex.net.br/}}
\end{figure}

\end{frame}

% ----------------- NOVO SLIDE --------------------------------
\begin{frame}{Participe dos grupos de discussão}

\begin{itemize}
  \item Tire dúvidas e ajude outros por meio do grupo de usuários LaTeX
  \url{https://groups.google.com/group/latex-br} (e-mail:
  \url{latex-br@googlegroups.com})
  
  \item Proponha melhorias, avise sobre falhas e faça sugestões sobre o abnTeX2
  no grupo dos desenvolvedores \url{https://groups.google.com/group/abntex2}
  (e-mail: \url{abntex2@googlegroups.com});
\end{itemize}

Participe também da comunidade abnTeX2 no Google Plus
\url{https://plus.google.com/u/0/communities/105202176004387477100}.

\end{frame}

% ----------------- NOVO SLIDE --------------------------------
\section{Resultados}

\begin{frame}
\frametitle{ABNT}
\framesubtitle{Normas para trabalhos acadêmicos}

Para adequar seus documentos acadêmicos com as normas ABNT, utilize:
\begin{enumerate}
 \item \citeonline{NBR14724:2011}: Esta Norma especifica os princípios gerais
 para a elaboração de trabalhos acadêmicos (teses, dissertações e outros),
 visando sua apresentação à instituição (banca, comissão examinadora de
 professores, especialistas designados e/ou outros).
 
 \item \citeonline{NBR6028:2003}: Esta Norma estabelece os requisitos para
 redação e apresentação de resumos.
 
 \item \citeonline{NBR6024:2012}: Esta Norma especifica os princípios gerais
 para de um sistema de numeração progressiva das seções de um documento, de
 modo a expor numa seqüência lógica o inter-relacionamento da matéria e a
 permitir sua localização.
 
 \item \citeonline{NBR10520:2002}: Esta Norma especifica as características
 exigíveis para a apresentação de citações em documentos.
\end{enumerate}

\end{frame}

% ----------------- NOVO SLIDE --------------------------------
\begin{frame}
\frametitle{abnTeX2}
\framesubtitle{Usando a suíte abnTeX2}

Consulte \citeonline{abntex2-wiki-como-customizar} para customizações do abnTeX2.
\vspace{0.5cm}

Os documentos \citeonline{abntex2modelo-artigo},
\citeonline{abntex2modelo-relatorio} e \citeonline{abntex2modelo} tratam dos
principais trabalhos acadêmicos e suas aplicações ao TeX.
\vspace{0.5cm}

Para orientações sobre as citações e as referências com o abnTeX2, consulte
\citeonline{abntex2cite} e \citeonline{abntex2cite-alf}.
\vspace{0.5cm}

\end{frame}

% ----------------- NOVO SLIDE --------------------------------
\section{Referências}

% --- O comando \allowframebreaks ---
% Se o conteúdo não se encaixa em um quadro, a opção allowframebreaks instrui 
% beamer para quebrá-lo automaticamente entre dois ou mais quadros,
% mantendo o frametitle do primeiro quadro (dado como argumento) e acrescentando 
% um número romano ou algo parecido na continuação.

\begin{frame}[allowframebreaks]{Referências}
\bibliography{abntex2-modelo-references}
\end{frame}

% ----------------- FIM DO DOCUMENTO -----------------------------------------
\end{document}